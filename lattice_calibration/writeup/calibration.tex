\documentclass[11pt,letter]{article}
\usepackage[top=0.65in,bottom=0.9in,left=0.85in,right=0.85in]{geometry}

%\def\baselinestretch{1.25}
\def\baselinestretch{1.0}

\usepackage[greek, english]{babel}
\usepackage{multicol}

\usepackage{graphicx}
\usepackage[export]{adjustbox}


% The use of the times package forces the use of the type-1 times
% roman font, but the times roman font does not look nice.
% Besides the times roman font still does not print correctly on
% the dopy printer.
%\usepackage{times}


\usepackage{fancyhdr}
\usepackage{amsmath}
\usepackage{amssymb}
\usepackage{bm}
\usepackage{bbold}
\usepackage{parskip}

\newcommand{\bv}[1]{\ensuremath{\bm{#1}}}
\newcommand{\vo}{\ensuremath{V_{0}}}
\newcommand{\bvo}{\ensuremath{\bv{V}_{0}}}
\newcommand{\er}{\ensuremath{E_{R}}}
\newcommand{\Lc}{\ensuremath{L_{\mathrm{c}}}}
\newcommand{\dsig}[1]{\ensuremath{ \frac{ d\,\sigma_{#1} }{d\,\Omega} }}
\newcommand{\dbl}{\ensuremath{ \uparrow\! \downarrow \, }}
\newcommand{\spup}{\ensuremath{ \uparrow }}
\newcommand{\spdn}{\ensuremath{ \downarrow}}

\newcommand{\pin}{\ensuremath{ P_{\text{I}}} }
\newcommand{\pret}{\ensuremath{ P_{\text{R}}} }
\newcommand{\win}{\ensuremath{ w_{\text{I}}} }
\newcommand{\wret}{\ensuremath{ w_{\text{R}}} }

\begin{document}

\section{Dipole potentials for lithium } 

The depth (in $\mu$K) for a 1064~nm beam of power $P$ (in mW)  and beam waist
$w$ (in $\mu$m) is given by
\begin{equation}
  U  =  u \frac{ P }{w^{2}}  
\end{equation}
where 
\begin{equation}
   u  = 38.709   \\ 
   %u_{532} & = 39.461   \\ 
\end{equation} 

The recoil energy for a 1064 nm photon is $E_{R}=$1.4~$\mu$K.

\section{Calibration of lattice depth}

To calibrate the lattice depth we perform lattice phase modulation
spectroscopy.  We do so by modulating the frequency of the lattice AOM.  This
has the effect of shaking the lattice wells back and forth.  The lattice depth
is given by 
\begin{equation}
 V_{0}  =  \frac{4u}{E_{R}}\frac{ \sqrt{ \pin \pret } }{ \win \wret}  
\end{equation}
where the subscripts I, R stand for input and retro respectively. We
define the retro factor $r$ as $\pret = r\pin$, so we have
\begin{equation}
 V_{0}  =  \frac{4u\pin}{E_{R}}\frac{ \sqrt{  r } }{ \win \wret}  
\end{equation}


\section{Measurement of radial frequency in lattice configuration} 

{\bf Lithium mass}:   The radial frequency is determined by the potential
energy profile and the mass of lithium.   Below we devise a convenient unit
for the lithium mass:
\begin{equation}
 m  = 6\,\text{AMU} =  \frac{6 h}{0.4\,\mu\text{m}^{2} } = 
 \frac{6\times48\,\mu\text{K} / \text{MHz}}{0.4\,\mu\text{m}^{2} }
  = 7.2\text{e-4} \, \frac{ \mu\text{K} }{ \mu\text{m}^{2}\,\, \text{kHz}}
\end{equation}

With the rotators in lattice configuration we can make a radial frequency
measurement.    The square of this radial frequency is given by
\begin{equation}
  \nu_{Lr}^{2} =  \frac{ u }{ m \pi^{2}  }  
   \left( \frac{\pret}{\wret^{4} }
      + \frac{ \sqrt{ \pret \pin }}{ \win \wret^{3}} 
      +  \frac{\pin}{\win^{4}}
      + \frac{ \sqrt{ \pret \pin }}{ \wret \win^{3} }
   \right) 
\end{equation}

Using the retro factor, we have 
\begin{equation}
  \nu_{Lr}^{2} =  \frac{ u \pin }{ m \pi^{2}  }  
   \left( \frac{r}{\wret^{4} }
      + \frac{ \sqrt{ r }}{ \win \wret^{3}} 
      +  \frac{1}{\win^{4}}
      + \frac{ \sqrt{ r }}{ \wret \win^{3} }
   \right) 
\end{equation}

\section{Measurement of radial frequency in dimple configuration}

With the rotators in dimple configuration we can make a radial frequency
measurement.  The square of this radial frequency is given by 
\begin{equation} 
   \nu_{Dr}^{2} = \frac{ u \pin }{ m \pi^{2} } \left( 
     \frac{ 1 }{ \win^{4} } + \frac{ r }{ \wret^{4} } 
    \right)
\end{equation} 

\section{Combining all the above}  

Using the results from the sections above we have 
\begin{equation}
\begin{split} 
 \frac{V_{0}}{\pin}    \frac{E_{R}}{4u} & = \frac{ \sqrt{  r } }{ \win \wret}  \\
 \frac{\nu_{Lr}^{2}}{\pin}   \frac{ m \pi^{2}  }{ u }  &  =
   \left( \frac{r}{\wret^{4} }
      + \frac{ \sqrt{ r }}{ \win \wret^{3}} 
      +  \frac{1}{\win^{4}}
      + \frac{ \sqrt{ r }}{ \wret \win^{3} }
   \right) \\  
 \frac{ \nu_{Dr}^{2}}{\pin} \frac{ m \pi^{2} }{ u  } &  = 
   \left(  
     \frac{ 1 }{ \win^{4} } + \frac{ r }{ \wret^{4} } 
    \right)
\end{split}
\end{equation}

The numerical factors that show up are 
\begin{equation}
\begin{split}
   \frac{E_{R}}{4u } & = 9.04\text{e-3} \equiv x_{1}\\  
   \frac{m\pi^{2}}{u} & = 1.84\text{e-4} \equiv x_{2} 
\end{split}
\end{equation}

In the experiment we measure the lattice depth, the trap frequencies, and the
beam power, so the equations above are written as
\begin{equation}
\begin{split}
 \frac{\pin}{V_{0}} 
 \left[ \frac{\text{mW}}{E_{R}} \right]  
 & = x_{1} \frac{ \win \wret }{ \sqrt{r}}  \\
  & \\ 
 \frac{\pin}{\nu_{Lr} ^{2}} 
 \left[ \frac{\text{mW}}{\text{kHz}^{2}} \right]  
& = \frac{ x_{2} }{ 
   \left( \dfrac{r}{\wret^{4} }
      + \dfrac{ \sqrt{ r }}{ \win \wret^{3}} 
      +  \dfrac{1}{\win^{4}} 
      + \dfrac{ \sqrt{ r }}{ \wret \win^{3} }
   \right)} \\ 
 & \\
 \frac{\pin}{ \nu_{Dr}^{2} }
 \left[ \frac{\text{mW}}{\text{kHz}^{2}} \right]  
 & = \frac{ x_{2} }{
   \left(  
     \dfrac{ 1 }{ \win^{4} } + \dfrac{ r }{ \wret^{4} } 
    \right) } \\
\end{split}
\end{equation}

\bibliographystyle{osa}
\bibliography{calibration}

\end{document}




